\subsection{Notations}
The utilities can be compiled and run on any POSIX compliant system.
Although not tested on Windows, they should work there as well. In order
to be able to make the utilities, you should have run make in the main directory,
so that an libefsl.a has been made. These utilities link against that library.

You can run the binaries without arguments to get a short help, which is repeated
here in the 'usage` section of the utilities.

All utilities take an image, or a device as their first argument. If the image/device
is a partition, and hence does not contain a partition table, it will be recognised and treated
as such.
If a partition table is found, the first partition with a FAT signature will be used.

Arguments between $<>$ brackets are mandatory, arguments between $[]$ brackets are optional.


\subsection{cpo}
    \subsubsection*{Usage}
	cpo $<$fs$>$ $<$file\_read$>$ $<$local\_write$>$ [bufsize]
    \subsubsection*{Operation}
	Cpo copies a file residing in a fat filesystem to your local file system. It copies
	outwards. The first argument \code{fs} is the image or device on which the source
	filesystem resides. The second argument (\code{file\_read} is the path and filename
	of the sourcefile in \code{fs}. The third argument (\code{local\_write}) is the
	local file to which you want to write the contents.

	The fourth, optional, arguments (\code{bufsize}) says cpo what size of copy buffers
	to use. This operates the same as dd. This is used in testing the library, the
	default value is 4096 bytes.

\subsection{cpi}
    \subsubsection*{Usage}
	cpi $<$fs$>$ $<$local\_read$>$ $<$file\_write$>$ [bufsize]
    \subsubsection*{Operation}
	Cpi copies a file from your local filesystem to a fat filesystem. It copies inwards.
        The first argument \code{fs} is the image or device on which the source
        filesystem resides. The second argument (\code{local\_read} is the path and filename
        of the sourcefile on your local filesystem. The third argument (\code{file\_write}) is the
        path and filename to which you want to write the contents.

        Cpi will bail out when the destination target already exists.

        The fourth, optional, arguments (\code{bufsize}) says cpi what size of copy buffers
        to use. This operates the same as dd. This is used in testing the library, the
        default value is 4096 bytes.

\subsection{cpa}
    \subsubsection*{Usage}
	cpa $<$fs$>$ $<$local\_read$>$ $<$file\_append$>$ [bufsize]
    \subsubsection*{Operation}
	Cpa copies a file from your local filesystem to a fat filesystem. It copies inwards.
        The first argument \code{fs} is the image or device on which the source
        filesystem resides. The second argument (\code{local\_read} is the path and filename
        of the sourcefile on your local filesystem. The third argument (\code{file\_append}) is the
        path and filename to which you want to write the contents.

        Cpa will append to the destination target, or create it if it doesn't exist yet.

        The fourth, optional, arguments (\code{bufsize}) says cpi what size of copy buffers
        to use. This operates the same as dd. This is used in testing the library, the
        default value is 4096 bytes.

\subsection{list}
    \subsubsection*{Usage}
	list $<$fs$>$ $<$directory$>$
    \subsubsection*{Operation}
	List will give a listing of existing files on a given filesystem \code{fs}, that
	reside in the directory \code{directory}. To get the contents of the root directory,
	give a forward slash (/) as the directory argument.

\subsection{mkdir}
    \subsubsection*{Usage}
	mkdir $<$fs$>$ $<$dirname$>$
    \subsubsection*{Operation}
	Mkdir will create a new directory in the filesystem \code{fs}. You must provide the full path
	of the new directory as an argument.

\subsection{rmfile}
    \subsubsection*{Usage}
	rmfile $<$fs$>$ $<$filename$>$
    \subsubsection*{Operation}
	Rmfile will delete a file from a filesystem \code{fs}.

