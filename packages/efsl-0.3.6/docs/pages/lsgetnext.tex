\subsubsection*{Purpose}
This function fetches the next valid file in the current directory and copies
all relevant information to \code{dirlist->currentEntry}.
\subsubsection*{Prototype}
\code{esint8 ls\_getNext(DirList *dlist);}
\subsubsection*{Arguments}
Objects passed to \code{ls\_getNext}:
\begin{itemize}
	\item{\code{dlist}: pointer to a DirList object}
\end{itemize}
\subsubsection*{Return value}
This function will return 0 when it has found a next file in the directory, and
was successful in copying it to \code{dirlist->currentEntry}. It will return -1
when there are no more files in the directory.

\subsubsection*{Example}
To browse through a directory you should first open it with \code{ls\_openDir} and
then you can call \code{ls\_getNext} in a loop to iterate through the files. Please
note that they are unsorted.
\lstset{numbers=left, stepnumber=1, numberstyle=\small, numbersep=5pt, tabsize=4}
\begin{lstlisting}
#include "efs.h"
#include "ls.h"

	void main(void)
{
		EmbeddedFileSystem efsl;
		DirList list;

		/* Initialize efsl */
		if(efs_init(&efsl,"/dev/sda")!=0){
			DBG((TXT("Could not initialize filesystem (err \%d).\n"),ret));
			exit(-1);
}

		/* Open the directory */
		ls_openDir(list,&(efsl.myFs),"/usr/bin/");

		/* Print a list of all files and their filesize */
		while(ls_getNext(list)==0){
			DBG((TXT("%s (%li bytes)\n"),
			list->currentEntry.FileName,
			list->currentEntry.FileSize));
		}

		/* Correctly close the filesystem */
		fs_umount(&efs.myFs);
}
\end{lstlisting}

Please note that it is not required to close this object, if you wish to switch
to another directory you can just call \code{ls\_openDir} on the object again.
